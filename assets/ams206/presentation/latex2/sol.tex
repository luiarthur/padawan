\documentclass[12pt]{article}
\usepackage{fullpage,amssymb}
\oddsidemargin .2 in
\evensidemargin .2 in
\topmargin 0 in
\textwidth 6.1 in
\textheight 8.3 in


\def\e#1{{\rm e}^{#1}}
\def\exp#1{{\rm exp}{#1}}

\def\frac#1#2{{{#1}\over{#2}}}
\def\binom#1#2{{{#1}\choose{#2}}}
\def\spot{$\bullet$\hspace{0.1cm}}

\def\T{\mathbf{T}}
\def\Y{\mathbf{Y}}
\def\m{\mathbf{m}}
\def\M{\mathbf{M}}
\def\C{\mathbf{C}}
\def\Q{\mathbf{Q}}
\def\A{\mathbf{A}}
\def\S{\mathbf{S}}
\def\ept{\epsilon_t}
\def\ei{\epsilon_i}
\def\pj{\phi_j}
\def\xtj{x_{t-j}}
\def\r#1{\rho_#1}
\def\omt{\omega_t}
\def\AR#1{{\rm AR}$(#1)$}
\def\be {\hbox{\bg\char'014}}
\def\de {\hbox{\bg\char'016}}
\def\ee {\hbox{\bg\char'021}}
\def\la {\hbox{\bg\char'025}}
\def\mm {\hbox{\bg\char'026}}
\def\nn {\hbox{\bg\char'027}}
\def\xx {\hbox{\bg\char'030}}
\def\pp {\hbox{\bg\char'031}}
\def\rh {\hbox{\bg\char'032}}
\def\ps {\hbox{\bg\char'040}}
\def\bo   {\hbox{\bf 0}}
\def \f  {\hbox{\bf f}}
\def \o  {\hbox{\bf o}}
\def \l  {\hbox{\bf l}}
\def \L  {\hbox{\bf L}}
\def \O  {\hbox{\bf O}}
\def \r  {\hbox{\bf r}}
\def \y  {\hbox{\bf y}}
\def \l  {\hbox{\bf l}}
\def \e  {\hbox{\bf e}}
\def \h  {\hbox{\bf h}}
\def \j  {\hbox{\bf j}}
\def \s  {\hbox{\bf s}}
\def \b  {\hbox{\bf b}}
\def \x  {\hbox{\bf x}}
\def \q  {\hbox{\bf q}}
\def \w  {\hbox{\bf w}}
\def \z  {\hbox{\bf z}}
\def \u  {\hbox{\bf u}}
\def\V   {\hbox{\bf V}}
\def\B   {\hbox{\bf B}}
\def\D   {\hbox{\bf D}}
\def\A   {\hbox{\bf A}}
\def\R   {\hbox{\bf R}}
\def\a   {\hbox{\bf a}}
\def\F   {\hbox{\bf F}}
\def\E   {\hbox{\bf E}}
\def\G   {\hbox{\bf G}}
\def\g   {\hbox{\bf g}}
\def\J   {\hbox{\bf J}}
\def\U   {\hbox{\bf U}}
\def\Z   {\hbox{\bf Z}}


\def\nl{\hfill\break}
\def\W   {\hbox{\bf W}}
\def\N   {\hbox{\bf N}}
\def\P   {\hbox{\bf P}}
\def\p   {\hbox{\bf p}}
\def\m   {\hbox{\bf m}}
\def\d   {\hbox{\bf d}}
\def\C   {\hbox{\bf C}}
\def\oned  {\hbox{\bf 1}}
\def\onednew {\hbox{\bigbf\char'061}}
\def\zerom {\hbox{\bf 0}}
\def\K   {\hbox{\bf K}}
\def\H   {\hbox{\bf H}}
\def\U   {\hbox{\bf U}}
\def\u   {\hbox{\bf u}}
\def\v   {\hbox{\bf v}}
\def\I   {\hbox{\bf I}}
\def\X   {\hbox{\bf X}}

\def\maE{\mbox{\boldmath$\mathcal E$}}
\def\bbrac{\mbox{\boldmath$[$}}
\def\bbracc{\mbox{\boldmath$]$}}

\def\epbold{\mbox{\boldmath$\epsilon$}}
\def\etabold{\mbox{\boldmath$\eta$}}
\def\ombold{\mbox{\boldmath$\omega$}}
\def\xin{\mbox{\boldmath$\xi$}}
\def\mun{\mbox{\boldmath$\mu$}}
\def\ph{\mbox{\boldmath$\phi$}}
\def\rh{\mbox{\boldmath$\rho$}}
\def\bet{\mbox{\boldmath$\beta$}}
\def\al{\mbox{\boldmath$\alpha$}}
\def\et{\mbox{\boldmath$\eta$}}
\def\nub{\mbox{\boldmath$\nu$}}
\def\np{\vfill\break}
\def\sline{\item{--}}
\def\ce#1{\centerline{#1}}\def\th{\mbox{\boldmath$\theta$}}
\def\om{\mbox{\boldmath$\omega$}}
\def\ga{\mbox{\boldmath$\gamma$}}
\def\del{\mbox{\boldmath$\delta$}}
\def\lam{\mbox{\boldmath$\lambda$}}
\def\ep{\mbox{\boldmath$\epsilon$}}
\def\n{\mbox{\boldmath$\nu$}}
\def\Ph{\mbox{\boldmath$\phi$}}
\def\Ps{\mbox{\boldmath$\psi$}}
\def\PS{\mbox{\boldmath$\Psi$}}
\def\TH{\mbox{\boldmath$\Theta$}}
\def\OM{\mbox{\boldmath$\Omega$}}
\def\PH{\mbox{\boldmath$\Phi$}}
\def\bzero{\mbox{\boldmath$0$}}
\def\Sig{\mbox{\boldmath$\Sigma$}}
\def\UP{\mbox{\boldmath$\Upsilon$}}
\def\GA{\mbox{\boldmath$\Gamma$}}
\def\DE{\mbox{\boldmath$\Delta$}}

\def\no{\noindent}

% Arthur
\usepackage{mathtools}
\usepackage{bm}
\def\ds{\displaystyle}
\def\beginmyfig{\begin{figure}[h]\center}
\def\endmyfig{\end{figure}}



\begin{document}
\title{Solutions to Problem 7 \& 11\\
Homework Assignment 2 \\
{\small {\bf AMS 206B, WINTER 2016 }} \\
}

\author{Prepared by: Sharmistha Guha and Arthur Lui}

\maketitle


\section*{Problem 7, HW 2}
\begin{align*}
  L(\theta-\theta(x)) = e^{c(\theta-\theta(x))} - c(\theta-\theta(x)) - 1\\
  = e^{c(\theta-\theta(x))} - c(\theta-\theta(x)) - 1\\
\end{align*}



\section*{Problem 11, HW 2}
Given that $X_1, X_2$ are two independent observations from
\begin{equation}
P(X=\theta-1|\theta)=P(X=\theta+1|\theta)=\frac{1}{2}
\end{equation}
where $\theta$ is an integer. \\
We are provided with the 0 -1 Loss Function
$$
L(\theta,\theta(X_1,X_2))=\left\{\begin{array}{c}
1\:\:\:\:\:\mbox{if}\:\:\theta(X_1,X_2)\neq \theta\\
0\:\:\:\:\:\:\:\:\:\:\:\:\:\:\:\:\:\:\:\:\:\:\:\:\:\:\:\:\mbox{o.w.}\\
\end{array}
\right.
$$

\textbf{(a)} \\(i) We proceed to find the risk of the estimator $\theta_0(X_1,X_2)=\frac{X_1+X_2}{2}$. \\
Let $\X=(X_1,X_2)'$
\begin{eqnarray}
  R(\theta,\theta_0(\X)) &=& \sum_{x_1\in\{\theta-1,\theta+1\}}\sum_{x_2\in\{\theta-1,\theta+1\}}L(\theta,\theta(x_1,x_2))P(X_1=x_1,X_2=x_2)\nonumber\\
&=& 1. P(X_1=\theta-1,X_2=\theta-1)+1.P(X_1=\theta+1,X_2=\theta+1)\nonumber\\
&+& 0.P(X_1=\theta-1,X_2=\theta+1)+0.P(X_1=\theta+1,X_2=\theta-1)\nonumber\\
&=& P(X_1=\theta-1)P(X_2=\theta-1)+P(X_1=\theta+1)P(X_2=\theta+1)\nonumber\\
&=& \frac{1}{2}.\frac{1}{2}+\frac{1}{2}.\frac{1}{2}=\frac{1}{2}.
\end{eqnarray}
(ii) Now considering the second estimator $\theta_1(X_1,X_2)=X_1+1$.
\begin{eqnarray}
R(\theta,\theta_1(\X) &=& \sum_{x_1\in\{\theta-1,\theta+1\}}\sum_{x_2\in\{\theta-1,\theta+1\}}L(\theta,\theta(x_1,x_2))P(X_1=x_1,X_2=x_2)\nonumber\\
&=& 1. P(X_1=\theta+1,X_2=\theta+1)+1.P(X_1=\theta+1,X_2=\theta-1)\nonumber\\
&+& 0.P(X_1=\theta-1,X_2=\theta+1)+0.P(X_1=\theta-1,X_2=\theta-1)\nonumber\\
&=& P(X_1=\theta+1)P(X_2=\theta+1)+P(X_1=\theta+1)P(X_2=\theta-1)\nonumber\\
&=& \frac{1}{2}.\frac{1}{2}+\frac{1}{2}.\frac{1}{2}=\frac{1}{2}.
\end{eqnarray}
The two estimators are the same as far as ``Frequentist Risk" is concerned.\\



\textbf{(b)} Need to find an estimator $\hat{\theta}(X_1,X_2)$ that minimizes Bayesian Expected Loss. \\

Bayesian Expected Loss $=E_{\theta|\X}(L(\theta,\theta(\X)))$
\begin{eqnarray}
E_{\theta|\X}(L(\theta,\theta(\X))) &=& 1. P_{\theta|\X}(\theta\neq \theta(\X))+0. P_{\theta|\X}(\theta=\theta(\X))\\
&=& P_{\theta|\X}(\theta\neq \theta(\X))=1-P_{\theta|\X}(\theta=\theta(\X)).
\end{eqnarray}
In order to minimize the Bayesian Expected Loss, we need to maximize \\
$P_{\theta|\X}(\theta=\theta(\X))$. \\
This implies
$\hat{\theta}(\X)=arg\max\limits_{\theta} P_{\theta|\X}$. \\

Let the prior distribution of $\theta$ be $\p(\cdot)$. 
Consider the following cases:\\

\emph{Case 1}: $X_1\neq X_2$ \\

$P(\theta=\frac{X_1+X_2}{2}|X_1,X_2)=1$. \\So, the mode of the posterior distribution is $\frac{X_1+X_2}{2}$. \\Therefore, when $X_1\neq X_2$, $\hat{\theta}(\X)=\frac{X_1+X_2}{2}$. \\

\emph{Case 2:} $X_1=X_2$\\

The posterior distribution of $\theta$ has mass on $X_1+1$ and $X_1-1$ and the posterior p.m.f is given by
\begin{equation}
P(\theta=X_1+1|\X)=\frac{\p(\theta=X_1+1)}{\p(\theta=X_1+1)+\p(\theta=X_1-1)}
\end{equation}
and 
\begin{equation}
P(\theta=X_1-1|\X)=\frac{\p(\theta=X_1-1)}{\p(\theta=X_1+1)+\p(\theta=X_1-1)}.
\end{equation}


Thus the mode of $\theta|\X$ is ($X_1+1$) if $\p(\theta=X_1+1)\geq \p(\theta=X_1-1)$. Also, the mode of $\theta|\X$ is ($X_1-1$) if $\p(\theta=X_1+1) < \p(\theta=X_1-1)$. \\
Hence
$$
\hat{\theta}(X_1,X_2)=\left\{\begin{array}{c}
\frac{X_1+X_2}{2}, \:\:\:\:\:\:\:\:\:\:\:\:\:\:\mbox{If}\:\:\:\:\:\:\:\:\:\:\:\:\:\:\:\:\:\:\:X_1\neq X_2\\
X_1+1,\:\:\mbox{If}\:\:X_1=X_2\:\:\mbox{and}\:\:\p(\theta=X_1+1)\geq \p(\theta=X_1-1)\\
X_1-1,\:\:\mbox{If}\:\:X_1=X_2\:\:\mbox{and}\:\:\p(\theta=X_1+1)<\p(\theta=X_1-1)\\
\end{array}
\right.
$$

\end{document}
